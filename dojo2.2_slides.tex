\documentclass{beamer}
\usepackage{hyperref}
\definecolor{links}{HTML}{2A1B81}
\hypersetup{colorlinks,linkcolor=,urlcolor=links}

\usepackage{verbatim}

\title{Haskell Dojo 2}
\subtitle{Taking it to practice}
\date{\today}
\author{Janne Haskell}

\begin{document}

\frame{\titlepage}

%-----------------------------------------------------------------
\begin{frame}[fragile]

\frametitle{Haskell Dojo 2.2}

We will incrementally work on an example project that demonstrates:

\begin{itemize}
  \item Basic project setup and compilation
  \item Parsing JSON
  \item I/O (working in the IO monad, networking)
  \item Monad basics
  \item Handling state in a pure language (game state, state monad)
\end{itemize}

\end{frame}
%-----------------------------------------------------------------

\begin{frame}[fragile]
\frametitle{Resources}

Links:

\begin{itemize}
  \item \href{https://github.com/nurpax/haskell-dojo/wiki/Haskell-Dojo-\%232}{Dojo wiki}
  \item \href{https://github.com/nurpax/haskell-dojo/wiki/Further-reading}{Further reading}
\end{itemize}

\end{frame}

%-----------------------------------------------------------------
\begin{frame}[fragile]

\frametitle{Today}

Today we will be looking at:

\begin{itemize}
  \item Homework solutions
  \item Working in the IO monad
  \item Networking
  \item Concurrency basics (forkIO, MVars)
\end{itemize}

\end{frame}
%-----------------------------------------------------------------



%-----------------------------------------------------------------
\begin{frame}[fragile]
\frametitle{Homework}

\textbf{Exercise 1}: Generalize \verb+countFoos+ to handle \verb+JSONDict+'s.
E.g., how to count foos when the input contains nested arrays and
dicts.

\bigskip

\begin{verbatim}
countFoos :: JSON -> Int
countFoos json = go json 0
  where
    go (JSONString "foo") acc = acc+1
    go (JSONString _)     acc = acc
    go (JSONInt _)        acc = acc
    go (JSONList xs)      acc = foldr go acc xs
\end{verbatim}

\end{frame}
%-----------------------------------------------------------------

%-----------------------------------------------------------------
\begin{frame}[fragile]
\frametitle{Homework}

\textbf{Solution 1}.

\bigskip

\begin{verbatim}
countFoosGen :: JSON -> Int
countFoosGen json = go json 0
  where
    go (JSONString "foo") acc = acc+1
    go (JSONString _)     acc = acc
    go (JSONInt _)        acc = acc
    go (JSONList xs)      acc = foldr go acc xs
    go (JSONDict m)       acc = M.foldr go acc m
\end{verbatim}

\end{frame}
%-----------------------------------------------------------------


%-----------------------------------------------------------------
\begin{frame}[fragile]
\frametitle{Homework}

\textbf{Exercise 2}: Implement a function \verb+takeInts :: JSON -> [Int]+
which takes JSON as input and concatenates the JSONInt elements into a
list of Ints ([Int]).

\bigskip
Example input and output:

{\tiny
\begin{verbatim}
intInputs :: [String]
intInputs =
intInputs :: [String]
intInputs =
  [ "[1, 2, 3, 4, 5, 6]"
  , "[1, [2, 3.1, 3, 4, 5, 6], [[7, 8]]]"
  , "[1, {\"foo\":{\"foo\":[\"foo\", [2, 3, 4, 5]]}}, \"foo\", \"bar\"]"
  , "[6, [[2, [1, [4, [3, 5]]]], [7,[9, [8]]]]]"
  ]

>  mapM_ (print . either (error . show) takeInts . Y.parseJson) intInputs
[1,2,3,4,5,6]
[1,2,3,4,5,6,7,8]
[1,2,3,4,5]
[6,2,1,4,3,5,7,9,8]
\end{verbatim}
}
\end{frame}
%-----------------------------------------------------------------


%-----------------------------------------------------------------
\begin{frame}[fragile]
\frametitle{Homework}

\textbf{Solution 2}.

\bigskip

\begin{verbatim}
takeInts :: Y.JSON -> [Int]
takeInts json = go json []
  where
    go (Y.JSONInt n)    acc = n : acc
    go (Y.JSONString _) acc = acc
    go (Y.JSONFloat _)  acc = acc
    go (Y.JSONList xs)  acc = foldr go acc xs
    go (Y.JSONDict xs)  acc = M.foldr go acc xs
\end{verbatim}

\end{frame}
%-----------------------------------------------------------------


%-----------------------------------------------------------------
\begin{frame}[fragile]
\frametitle{Homework}

\textbf{Exercise 3}: Implement a function
\verb+buckets :: JSON -> M.Map String Int+ which computes a histogram
of string (JSONString) elements in the input.

\bigskip
Example input and output:

{\tiny
\begin{verbatim}
testInput :: String
testInput = "[[\"bar\", \"foo\"], \"foo\",[\"bar\", \"foo\"],\"bar\",\"foo\",\"baz\",3,\"foo\"]"

> either (error . show) (print . buckets) $ parseJson testInput
fromList [("bar",3),("baz",1),("foo",5)]

\end{verbatim}
}
\end{frame}
%-----------------------------------------------------------------


%-----------------------------------------------------------------
\begin{frame}[fragile]
\frametitle{Homework}

\textbf{Solution 3}.

\bigskip

\begin{verbatim}
buckets :: JSON -> M.Map String Int
buckets json = go json M.empty
  where
    go (JSONString s) acc = M.insertWith (+) s 1 acc
    go (JSONInt _)    acc = acc
    go (JSONList xs)  acc = foldr go acc xs
    go (JSONDict m)   acc = M.foldr go acc m
\end{verbatim}

\end{frame}
%-----------------------------------------------------------------


%-----------------------------------------------------------------
\begin{frame}[fragile]
\frametitle{Homework}

\textbf{Exercise 4}: Implement \verb+insert+ and \verb+toList+ using the below definitions:

{\small
\begin{verbatim}
-- Binary tree with Int values
data Tree = Tip | Node Int Tree Tree
          deriving (Show)

empty :: Tree
empty = Tip

insert :: Int -> Tree -> Tree
insert = -- TODO!!

-- Return the contents of a sorted binary tree in-order
toList :: Tree -> [Int]
toList = -- TODO
\end{verbatim}
}
\end{frame}
%-----------------------------------------------------------------

%-----------------------------------------------------------------
\begin{frame}[fragile]
\frametitle{Homework}

\textbf{Solution 4}.

\bigskip

\begin{verbatim}
insert :: Int -> Tree -> Tree
insert v Tip = Node v Tip Tip
insert v (Node v' left right)
  | v < v'    = Node v' (insert v left) right
  | otherwise = Node v' left (insert v right)

toList :: Tree -> [Int]
toList Tip = []
toList (Node v l r) = toList l ++ [v] ++ toList r
\end{verbatim}

\end{frame}
%-----------------------------------------------------------------

\end{document}
