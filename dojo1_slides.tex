\documentclass{beamer}

\usepackage{verbatim}

\title{Haskell Dojo 1}
\subtitle{A tour of Haskell}
\date{\today}
\author{Riku Salminen}

\begin{document}

\frame{\titlepage}

\begin{frame}[fragile]
\frametitle{Haskell Dojo}

\begin{itemize}
\item A quick tour of Haskell, not an in-depth course
\item Three parts: expressions, types and imperative programming
\item Learning by example and by trying it out
\item So let's get started with GHCi!
\item In your favorite text editor, type:
\end{itemize}

\begin{figure}[h!]
\caption{hellohaskell.hs}
\begin{verbatim}
main = putStrLn "Hello Haskell!"
\end{verbatim}
\end{figure}

\begin{itemize}
\item run: \verb+ghci hellohaskell.hs+
\item \verb :load  and \verb :reload .

\end{itemize}
\end{frame}


\begin{frame}[fragile]
\frametitle{Homework}

\begin{itemize}
\item Homework: 5 card draw poker
\item Homework review workshop in a few weeks
\item NOTE: Get shuffle from Haskell wiki
\end{itemize}
\end{frame}


\begin{frame}[fragile]
\frametitle{Haskell lexical structure}

\begin{itemize}
\item Whitespace sensitive indentation (like Python)
\item Identifiers lower camel case: \verb foo123 , \verb scarlettO'Hara , \verb x'' .
\item Modules and types upper camel: \verb Integer , \verb+Set Integer+, \verb Main .
\item Infix operators: \verb|+|, \verb|++|, \verb|==|, \verb|/=|, \verb|>>=|
\item Inline comment: \verb|x = 13  -- this is a comment|
\item Block comment: \verb|{- This is a multi-line comment -}|
\end{itemize}
\end{frame}

\begin{frame}[fragile]
\frametitle{Haskell expressions}

\begin{itemize}
\item Function application: \verb|mod 7 2|
\item Infix expressions: \verb|x + y * z|
\item Application binds tightest: \verb|mod (x+1) 32 /= 0|
\item Conditional: \verb|if mod x 2 /= 0 then "Odd" else "Even"|
\item \verb if  is an expression (like \verb|x ? y : z|), not a statement
\item There's no \verb elseif , use guard syntax instead!
\item \verb|let x = 42 in mod x 12|
\item \verb|mod x 12 where x = 42|
\item $\lambda$ -functions: \verb|\x y -> x + y|
\item Typing hints: \verb|read "123" :: Integer|
\end{itemize}
\end{frame}


\begin{frame}[fragile]
\frametitle{Haskell functions}

\begin{itemize}
\item Type these in a Haskell file, not GHCi!
\item Ex. multiply and add: \verb|mad a b c = a + b * c|
\item Same with $\lambda$ expression: \verb|mad = \a b c -> a + b * c|
\item Many definitions with pattern matching:
\end{itemize}

\begin{verbatim}
pow _ 0 = 1
pow x y =
    if mod y 2 == 0
        then square (pow x (div y 2))
        else x * pow x (y-1)
    where
    square x = x * x
\end{verbatim}

\end{frame}

\begin{frame}[fragile]
\frametitle{Haskell function magic}

\begin{itemize}
\item Function application: \verb|f $ x + 1| equals \verb|f (x+1)|
\item Function composition $(f \circ g) (x)$: \verb|(f . g) x| equals \verb|f (g x)|
\item Idiomatic way: \verb|square . pow x . flip div 2 $ y|
\item Partial application: \verb|powerOfTwo = pow 2|
\item Would you like some Curry with your code?
\item See also: \verb curry , \verb uncurry .
\end{itemize}
\end{frame}

\begin{frame}[fragile]
\frametitle{Haskell types}

\begin{itemize}
\item Haskell is statically typed with type inference.
\item Compiler "guesses" types, gives errors on ambiguity
\item Sum types implemented with tagged unions
\item \verb+data Passenger = Person String Int | Animal Int+ .
\end{itemize}

\begin{verbatim}
describePassenger passenger =
    case passenger of
        (Person name age) -> name ++ ", age: " ++ show age
        (Animal weight) -> "Animal weight: " ++ weight

describePassenger (Person name age) =
    name ++ ", age: " ++ show age
describePassenger (Animal weight) =
    "Animal weight: " ++ weight
\end{verbatim}
\end{frame}

\begin{frame}[fragile]
\frametitle{Built-in types}

\begin{itemize}
\item Integers: \verb Integer , \verb Int , \verb Word16 .
\item Functions: \verb|Int -> Int -> Bool|
\item Tuples: \verb+("Spam", 13) :: (String, Int)+
\item Lists: \verb+[13, 42, 12] :: [Integer]+
\item \verb+String+'s are lists of \verb+Char+'s.
\item \verb+[("Spam", 13), ("Eggs", 42)] :: [(String, Int)]+
\item Lots of excellent data structures in standard library:
\item \verb+import qualified Data.Map as Map+
\item \verb+import qualified Data.Set as Set+
\item Load modules in GHCi: \verb|:m +Data.Map|
\end{itemize}
\end{frame}

\begin{frame}[fragile]
\frametitle{Advanced types}

\begin{itemize}
\item Data types can be recursive
\item \verb+data List a = Cons a (List a) | Nil+
\item \verb+data Tree a = Branch (Tree a) (Tree a) | Leaf a+
\item A more complete example:
\end{itemize}

\begin{verbatim}
data Suit = Hearts | Spades | Diamonds | Clubs 
    deriving (Show, Eq, Ord)

data Card =
    Card { suit :: Suit, value :: Int } |
    Joker
    deriving (Show, Eq, Ord)
\end{verbatim}

\begin{itemize}
\item \verb+Show, Eq, Ord+ are type classes: showable, equality comparable and ordered.
\end{itemize}

\end{frame}


\begin{frame}[fragile]
\frametitle{Fun with lists}

\begin{itemize}
\item Tail recursion modulo cons
\end{itemize}

\begin{verbatim}
plus1 list =
    if null list  -- null means List.isEmpty
        then []
        else ((head list) + 1) : plus1 (tail list)

plus1 [] = []
plus1 (x:xs) = (x+1) : plus1 xs
\end{verbatim}

\begin{itemize}
\item Somewhat unwieldy. That's why there's \verb map , \verb filter , \verb foldl .
\item \verb|map (+1) [1, 2, 3]| equals \verb|[2, 3, 4]|.
\item \verb|filter even [1..10]| equals \verb|[2,4,6,8,10]|.
\item \verb|foldl max 0 [5, 11, 1, 13, 2]| equals \verb|13|.
\item "Sum of squares of odd natural numbers less than 100"
\item \verb|foldl (+) 0 . map (^2) . filter odd $ [1..100]|
\item \verb+sum [x*x | x <- [1..100], odd x]+
\end{itemize}

\begin{verbatim}
\end{verbatim}

\end{frame}

\begin{frame}[fragile]
\frametitle{Imperative programming}

\begin{itemize}
\item Haskell has "first class statements". Statements are expressions, \textbf{not} the other way.
\item \verb do  is syntactic sugar for composing sequential statements
\item Compiler transforms to concrete notation using \verb >>= .
\item \verb|return 13|
\item \verb|let x = 13|
\item \verb|content <- readFile "/etc/hostname"|
\item \verb|do { line <- getLine ; putStrLn line }| equals \verb|getLine >>= \line -> putStrLn line| equals \verb|getLine >>= putStrLn|.
\item In GHCi, you're inside a \verb|do| block. "Fixed" in latest version.
\item A monad is a monoid in the category of endofunctors, what's the problem?
\end{itemize}
\end{frame}

\begin{frame}[fragile]
\frametitle{Guess the number}
\begin{verbatim}
doGuess :: Int -> Int -> IO Bool
doGuess _ 0 = return False
doGuess answer attempts = do
    line <- getLine
    let guess = read line
    if guess == answer
        then return True
        else do
            putStrLn (if guess < answer then "Lo" else "Hi")
            doGuess answer (attempts-1)

main :: IO ()
main = do
    answer <- randomRIO (1, 10)
    result <- doGuess answer 3
    putStrLn $ "You " ++ (if result then "Win" else "Lose")
\end{verbatim}
\end{frame}

\begin{frame}[fragile]
\frametitle{Mutable variables}

\begin{itemize}
\item Haskell has mutable variables, but they're not used often
\item Different types of variables with different kinds of semantics
\item The regular kind of variable is called \verb IORef  (\verb|Data.IORef|).
\item Create a new variable: \verb|var <- newIORef 13|
\item Write value: \verb|writeIORef var 42|
\item Read value: \verb|x <- readIORef var|
\item See also \verb modifyIORef  and \verb atomicModifyIORef .
\item Other kinds of variables: \verb MVar  (safe concurrent access), \verb TVar  (software transactional memory), \verb STRef  (inside pure code).
\end{itemize}
\end{frame}

\begin{frame}[fragile]
\frametitle{Roadmap}

\begin{itemize}
\item Next workshop in about a month
\item Parsing with Parsec
\item Concurrency with forkIO
\item Real World Haskell with examples (MickiBot 2.0)
\end{itemize}
\end{frame}

\end{document}
